\documentclass[12pt,t]{beamer}
\usepackage{graphicx}
\setbeameroption{hide notes}
\setbeamertemplate{note page}[plain]
\usepackage{listings}
\usepackage{datetime}
\usepackage{url}

% specifications for presenter mode
\beamerdefaultoverlayspecification{<+->}
\setbeamercovered{transparent}

\usepackage[english]{babel}
\usepackage[utf8x]{inputenc}

\usepackage{amstext}
%\usepackage{coloremoji}

\usepackage{graphicx}
\graphicspath{ {Figs/} }

% math shorthand
\usepackage{bm}
\usepackage{amsmath}
\usepackage{mathtools}
\newcommand{\D}{\mathcal{D}}
\newcommand{\E}{\mathbb{E}}
\newcommand{\F}{\mathcal{F}}
\newcommand{\X}{\mathcal{X}}
\newcommand{\lik}{\mathcal{L}}
\newcommand{\indep}{\rotatebox[origin=c]{90}{$\models$}}
\DeclarePairedDelimiterX{\infdivx}[2]{(}{)}{%
  #1\;\delimsize\|\;#2%
}
\newcommand{\infdiv}{D\infdivx}
\DeclarePairedDelimiter{\norm}{\lVert}{\rVert}
\DeclareMathOperator*{\argmin}{arg\,min}
\DeclareMathOperator*{\argmax}{arg\,max}

% Bibliography
\usepackage{natbib}
\bibpunct{(}{)}{,}{a}{}{;}
\usepackage{bibentry}
\nobibliography*

\input{header.tex}

%%%%%%%%%%%%%%%%%%%%%%%%%%%%%%%%%%%%%%%%%%%%%%%%%%%%%%%%%%%%%%%%%%%%%%
% end of header
%%%%%%%%%%%%%%%%%%%%%%%%%%%%%%%%%%%%%%%%%%%%%%%%%%%%%%%%%%%%%%%%%%%%%%

% title info
\title{\normalsize Robust Nonparametric Inference for Stochastic Interventions
  Under Multi-Stage Sampling}
\subtitle{\scriptsize for the UC Berkeley Biostatistics Seminar Series,\\ given
                      02 April 2018 \\[-10pt]
         }
\author{\href{https://nimahejazi.org}{Nima Hejazi}
       \\[-10pt]
       }
\institute{Group in Biostatistics \\
           University of California, Berkeley \\
           \href{https://www.stat.berkeley.edu/~nhejazi}
             {\tt \scriptsize \color{foreground}
               stat.berkeley.edu/\textasciitilde{}nhejazi
             }
           \\[4pt]
           \includegraphics[height=20mm]{Figs/seal-berkeley.png}
           \\[-12pt]
          }
\date{
  \href{https://nimahejazi.org}
      {\tt \scriptsize \color{foreground} nimahejazi.org}
  \\[-4pt]
  \href{https://twitter.com/nshejazi}
      {\tt \scriptsize \color{foreground} twitter/@nshejazi}
  \\[-4pt]
  \href{https://github.com/nhejazi}
      {\tt \scriptsize \color{foreground} github/nhejazi}
}

%%%%%%%%%%%%%%%%%%%%%%%%%%%%%%%%%%%%%%%%%%%%%%%%%%%%%%%%%%%%%%%%%%%%%%%%%%%%%%%%

\begin{document}

% title slide
{
\setbeamertemplate{footline}{} % no page number here
\frame{
  \titlepage

  \vspace{-1em}

  \centerline{\href{https://goo.gl/Vq6v5o}{\tt \scriptsize
                                           \underline{slides}: goo.gl/Vq6v5o}}
  \vspace{-1.5em}
  \vfill \hfill \includegraphics[height=6mm]{Figs/cc-zero.png} \vspace*{-0.5cm}

  \note{This slide deck is for a seminar-length talk (about 50 minutes) on a new
    approach to causal inference and nonparametric variable importance in the
    context of parameters defined as treatment shifts. Here, we introduce an
    additive treatment shift parameter, extensions for censored data (including
    a multiple double robustness property), new statistical software for
    applying our approach, and applications to a vaccine efficacy trial
    examining HIV. This talk was most recently given at a meeting of the
    Biostatistics Seminar Series at the University of California, Berkeley.

    Source: {\tt https://github.com/nhejazi/talk\_txshift} \\
    Slides: {\tt https://goo.gl/LAoDUJ} \\
    With notes: {\tt https://goo.gl/Vq6v5o}
}
}
}

%%%%%%%%%%%%%%%%%%%%%%%%%%%%%%%%%%%%%%%%%%%%%%%%%%%%%%%%%%%%%%%%%%%%%%%%%%%%%%%%

\begin{frame}[c]{Preview: Summary}
\only<1>{\addtocounter{framenumber}{-1}}

\begin{center}
\begin{itemize}
  \itemsep10pt
  \item The evaluation of vaccine efficacy is a high-impact scientific problem
    that leads to numerous statistical challenges.
  \item Stochastic interventions provide a flexible framework through which
    these statistical problems may be viewed from the perspective of causal
    inference.
  \item Standard targeted minimum loss-based estimation may be augmented to
    handle multi-stage sampling designs, like those common in efficacy trials.
  \item Statistical software is now readily available for deploying these types
    of techniques in a number of settings. We apply these methods in efficacy
    trials.
\end{itemize}
\end{center}

\note{We'll go over this summary again at the end of the talk. Hopefully, it
  will all make more sense then.
}

\end{frame}

%%%%%%%%%%%%%%%%%%%%%%%%%%%%%%%%%%%%%%%%%%%%%%%%%%%%%%%%%%%%%%%%%%%%%%%%%%%%%%%%

\begin{frame}[c]{Motivation: Let's meet the data}

\begin{center}
\begin{itemize}
  \itemsep10pt
  \item HIV Vaccine Trials Network (HVTN) 505 HIV-1 vaccine efficacy trial.
  \item 2504 participants, with all observed cases matched to controls after
    collection of endpoints of interest.
  \item Background quantities ($W$): sex, age, BMI, etc.
  \item Variables of interest ($A$): biomarkers of immune response (e.g., T-Cell
    response).
  \item Outcome of interest ($Y$): HIV-1 infection risk.
  \item \textbf{Question}: How would changes in the immune response profile
    impact risk of HIV-1 infection?
\end{itemize}
\end{center}

\note{
\begin{itemize}
  \itemsep10pt
  \item A vaccine effective at preventing HIV-1 acquisition would be a
    cost-effective and durable approach to halting the worldwide epidemic.
  \item Identifying vaccine-induced immune-response biomarkers that predict a
    vaccine's ability to protect individuals from HIV-1 infection is a high
    priority.
  \item The study was halted on 22 April 2013 due to absence of vaccine
    efficcacy. There was no significant effect of the vaccine on the primary
    infection end- point of HIV-1 infection between week 28 and month 24.
\end{itemize}
}

\end{frame}

%%%%%%%%%%%%%%%%%%%%%%%%%%%%%%%%%%%%%%%%%%%%%%%%%%%%%%%%%%%%%%%%%%%%%%%%%%%%%%%%

\begin{frame}[c]{Preventive Vaccines for HIV}

\begin{center}
\begin{itemize}
  \itemsep10pt
  \item Substantial heterogeneity is present in the genetic characteristics of
    HIV.
  \item Preventive HIV vaccines constructed using only several antigens (out of
    a great many).
  \item \textbf{Success:} Protect well against infection caused by virus strains
    \textit{similar} to the source strain.
  \item \textbf{Failure:} Don't protect against disease caused by strains
    antigenically \textit{dissimilar} to source strain.
\end{itemize}
\end{center}

\note{
\begin{itemize}
  \itemsep10pt
  \item HIV is a high-impact public health issue but numerous attempts to
    develop vaccines have met with only mild success.
  \item The complexity of the disease mechanism makes it quite challenging to
    study the numerous factors that contribute to a possible mitigation of
    infection risk.
\end{itemize}
}

\end{frame}

%%%%%%%%%%%%%%%%%%%%%%%%%%%%%%%%%%%%%%%%%%%%%%%%%%%%%%%%%%%%%%%%%%%%%%%%%%%%%%%%

\begin{frame}[c]{Sieve Analysis: A Brief History}

\begin{center}
\begin{itemize}
  \itemsep10pt
  \item The study of whether and how the efficacy of a vaccine varies with the
    virus' characteristics.
  \item Why ``sieve''? Vaccine as a barrier against select strains, but
    dissimilar strains break through.
  \item Identification of sieve effects guides decisions for future development
    of multivalent vaccines.
  \item Sieve analysis is usually performed within a \textit{competing risks}
    framework.
\end{itemize}
\end{center}

\note{
\begin{itemize}
  \itemsep10pt
  \item The reliance on competing risks leads to the use of nonparametric
    estimators like Aalen-Johnson or semiparametric methods like the Cox model.
  \item Within this framework, could evaluate instantaneous risks of infection
    (i.e., hazard) or cumulative incidence. The latter could be more interesting
    from a public health perspective.
\end{itemize}
}

\end{frame}

%%%%%%%%%%%%%%%%%%%%%%%%%%%%%%%%%%%%%%%%%%%%%%%%%%%%%%%%%%%%%%%%%%%%%%%%%%%%%%%%

\begin{frame}[c]{Immune Response and Vaccine Efficacy}
\begin{center}
\begin{itemize}
  \itemsep10pt
  \item A 12-color intracellular cytokine staining (ICS) assay was performed.
  \item Cryopreserved peripheral blood mononuclear cells were stimulated with
    synthetic HIV-1 peptide pools.
  \item Immune responses of interest were
    \begin{enumerate}
      \item Total magnitude of the $\text{CD4}^+$ T-cell response.
      \item COMPASS Env-specific $\text{CD4}^+$ T-cell polyfunctionality score.
      \item Total magnitude of the $\text{CD8}^+$ T-cell response.
      \item COMPASS Env-specific $\text{CD8}^+$ T-cell polyfunctionality score.
      \item $\text{CD4}^+$ and $\text{CD8}^+$ T-cell
        $\text{log}_{10}$-transformed total magnitude variables.
    \end{enumerate}
\end{itemize}
\end{center}

\note{
\begin{itemize}
  \itemsep10pt
  \item For a complete description of the immune responses of interests and how
    these were collected, consult the supplemental materials of HE Janes (2017).
  \item This class of data is difficult and expensive to collect, which begins
    to provide motivation for why it might be undesirable to restrict the types
    of analyses performed to classical semiparametrics.
  \item Such classical analyses severely restrict the scope of the scientific
    questions we're able to ask.
\end{itemize}
}
\end{frame}

%%%%%%%%%%%%%%%%%%%%%%%%%%%%%%%%%%%%%%%%%%%%%%%%%%%%%%%%%%%%%%%%%%%%%%%%%%%%%%%%%%%%

\begin{frame}[c]{Immune Reponse and Vaccine Efficacy}
\begin{center}
\begin{itemize}
  \itemsep10pt
  \item \textit{Goal:} Evaluate the immune response variables among vaccine
    recipients as predictors of HIV-1 infection.
  \item Cox proportional hazard models that account for case-control sampling
    design and adjust for the baseline covariates.
  \item $\lambda \left(t; Z = z \right) = \lambda_0(t) \exp\left(\beta^Tz
    \right), \quad t\geq 0$.
    \begin{itemize}
      \item Semiparametric overall.
      \item nonparametric in $\lambda_0$, parametric in $\beta$.
    \end{itemize}
  \item Corrections for multiple testing performed, with q-values below $0.20$
    considered significant.
\end{itemize}
\end{center}

\note{
  \begin{itemize}
    \item Principal components analysis (PCA) was used to discover unique immune
      response profiles among vaccine recipients.
    \item First and second principal components were associated with HIV-1
      infection using Cox proportional hazards regression models that account
      for the sampling design and baseline covariates.
    \item Logistic regression models with lasso penalty and weights to account
      for case-control sampling were used to identify the baseline covariates
      and immune response variables that best predict HIV-1 infection.
  \end{itemize}
}
\end{frame}

%%%%%%%%%%%%%%%%%%%%%%%%%%%%%%%%%%%%%%%%%%%%%%%%%%%%%%%%%%%%%%%%%%%%%%%%%%%%%%%%%%%%

\begin{frame}[c]{Motivation: Science Before Statistics}
\begin{center}
\begin{itemize}
  \itemsep10pt
  \item Cox model: assumption of proportional hazards.
  \item Such models are a matter of convenience: does $\hat{\beta}$ answer our
    scientific questions?
    \begin{itemize}
      \item Perhaps not.
    \end{itemize}
  \item Is consideration being given to whether the data could have been
    generated by a process that is consistent with the assumptions of the Cox
    model?
    \begin{itemize}
      \item Perhaps not.
    \end{itemize}
\end{itemize}
\end{center}

\note{
}
\end{frame}

%%%%%%%%%%%%%%%%%%%%%%%%%%%%%%%%%%%%%%%%%%%%%%%%%%%%%%%%%%%%%%%%%%%%%%%%%%%%%%%%%%%%

\begin{frame}[c]{Interlude: Causal Inference}

\begin{center}
\begin{enumerate}
  \itemsep10pt
  \item Motivation: ``We do not have knowledge of a thing until we have grasped
    its why, that is to say, its cause.'' --Aristotle
  \item Our question of interest concerns the manner in which changes in a given
    immune response profile affect risk of HIV-1 infection.
    \begin{itemize}
      \item This is a question of causality.
      \item How does \textit{intervening} on immune response profile cause
        changes in the risk of HIV-1 infection.
    \end{itemize}
  \item But how do we go about thinking about intervening on continuous
    quantities (e.g., immune response profile measures)?
  \item Classical causal parameters (e.g., ATE) are not well suited for
    answering these sorts of questions.
\end{enumerate}
\end{center}

\note{
}

\end{frame}

%%%%%%%%%%%%%%%%%%%%%%%%%%%%%%%%%%%%%%%%%%%%%%%%%%%%%%%%%%%%%%%%%%%%%%%%%%%%%%%%

\begin{frame}[c]{Causal Inference and Vaccine Efficacy}

\begin{center}
\begin{itemize}
  \itemsep10pt
  \item Consider observing $n$ individuals in a data structure of the form
    specified above.
  \item To formalize, consider $O = (W, A, Y) \sim P_0 \in \mathcal{M}$, where
    we make no assumptions on the statistical model containing $P_0$.
  \item For the treatment $A$, we would normally be limited to thinking about
    counterfactual means (i.e., $\E Y_a$ for $A = a$) or similar quantities.
  \item This requires specifying a particular value of the treatment (i.e.,
    $A = a$) under which to evaluate the outcome.
\end{itemize}
\end{center}

\note{
}

\end{frame}

%%%%%%%%%%%%%%%%%%%%%%%%%%%%%%%%%%%%%%%%%%%%%%%%%%%%%%%%%%%%%%%%%%%%%%%%%%%%%%%%

\begin{frame}[c]{Stochastic Treatment Regimes}

\begin{center}
\begin{itemize}
  \itemsep10pt
  \item Rather than a deterministic intervention, consider a shift of the
    treatment (i.e., instead of $A = a$, consider $A = a + \delta$).
  \item This is a far more flexible approach. We need not specify a given value
    of the treatment but rather a shift ($\delta$) of the treatment.
  \item In this setting, the effect of the intervention appears as $\E Y_{a +
    \delta} - \E Y_a$, where $A = a$ is simply the observed value of
    treatment.
  \item To compare with the linear model, the shift $\delta$ may be thought of
    as analogous to shifts in the slope of the regression line.
\end{itemize}
\end{center}

\note{
}

\end{frame}

%%%%%%%%%%%%%%%%%%%%%%%%%%%%%%%%%%%%%%%%%%%%%%%%%%%%%%%%%%%%%%%%%%%%%%%%%%%%%%%%

\begin{frame}[c]{Problems with Stochastic Interventions}

\begin{center}
\begin{itemize}
  \itemsep10pt
  \item Even though we employ a more flexible type of intervention, the common
    assumptions (and problems!) of causal inference still arise.
    \begin{itemize}
      \item Randomization: $A \indep Y \mid W$
      \item Positivity: $0 < P(A \mid W) < 1$ everywhere. The propensity score
        is bounded in $(0, 1)$.
    \end{itemize}
  \item To protect against positivity violations, a clever shifting mechanism:
    $d(a, w) = a + \delta$, if $a + \delta < u(w)$ and $d(a, w) = a$ otherwise.
  \item The shift $d(A, W)$ is now a function of the observed data, and the
    shift intervention ($a + \delta$) is only applied when there is support in
    the observed data.

\end{itemize}
\end{center}

\note{
}

\end{frame}


%%%%%%%%%%%%%%%%%%%%%%%%%%%%%%%%%%%%%%%%%%%%%%%%%%%%%%%%%%%%%%%%%%%%%%%%%%%%%%%%

\begin{frame}[c]{Parameters for Treatment Shifting}

\begin{center}
\begin{itemize}
  \itemsep10pt
  \item Let's consider a simple target parameter: the average treatment effect
    (ATE):
    \[
      \Psi(P) = \E{\bar{Q}(A,W) I(A \leq \delta)} + \E{\bar{Q}(A,W)
        I(A > \delta)}
    \]
  \item Assume \textit{piecewise smooth invertibility} of $d(a,w)$ in order to
    obtain a pathwise differentiability of the parameter.
  \item This makes semiparametric-efficient estimation in the nonparametric
    model possible when relying on stochastic interventions.
  \item The parameter now corresponds to our scientific question of interest:
    How does shifting immune response by an amount $\delta$ affect the risk of
    HIV-1 infection?
\end{itemize}
\end{center}

\note{
By allowing scientific questions to inform the parameters that we choose
to estimate, we can do a better job of actually answering the questions of
interest to our collaborators. Further, we abandon the need to specify the
functional relationship between our outcome and covariates; moreover, we
can now make use of advances in machine learning.
}

\end{frame}

%%%%%%%%%%%%%%%%%%%%%%%%%%%%%%%%%%%%%%%%%%%%%%%%%%%%%%%%%%%%%%%%%%%%%%%%%%%%

\begin{frame}[c]{Semiparametric-Efficient Estimation}

\begin{center}
\begin{itemize}
  \itemsep10pt
  \item Our parameter of interest is
    \[
      \Psi(P) = \E_P{\bar{Q}(d(A, W), W)}
    \]
  \item For which the efficient influence function (EIF) is
    \[
      D(P)(o) = H(a, w){y - \bar{Q}(a, w)} + \bar{Q}(d(a, w), w) - \Psi(P)
    \]
  \item The auxiliary covariate introduced (i.e., $H(a,w)$) may be expressed
    \[
      H(a,w) = I(a < u(w)) \frac{g_0(a - \delta \mid w)}{g_0(a \mid w)} + I(a
      \geq u(w) - \delta)
    \]
\end{itemize}
\end{center}

\note{
  The auxiliary covariate simplifies when the treatment is in the limits
  (conditional on $W$) --- i.e., for $A_i \in (u(w) - \delta, u(w))$, then we
  have $H(a,w) = \frac{g_0(a - \delta \mid w)}{g_0(a \mid w)} + 1$.
}

\end{frame}

%%%%%%%%%%%%%%%%%%%%%%%%%%%%%%%%%%%%%%%%%%%%%%%%%%%%%%%%%%%%%%%%%%%%%%%%%%%%

\begin{frame}[c]{Target Minimum Loss-Based Estimation}

\begin{center}
\begin{itemize}
  \itemsep10pt
  \item TMLEs provide semiparametric-efficient estimation and robust inference
    in nonparametric models.
  \item \textbf{Asymptotic linearity:}
    \[
      \Psi(P_n^*) - \Psi(P_0) = \frac{1}{n} \sum_{i = 1}^{n} IC(O_i) +
      o_P\left(\frac{1}{\sqrt{n}}\right)
    \]
  \item \textbf{Limiting distribution:}
    \[
      \sqrt{n}(\Psi_n - \Psi) \to N(0, Var(D(P_0)))
    \]
  \item \textbf{Statistical inference:}
    \[
      \Psi_n \pm z_{\alpha} \cdot \frac{\sigma_n}{\sqrt{n}}
    \]
\end{itemize}
\end{center}

\note{
Under the additional condition that the remainder term $R(\hat{P}^*, P_0)$
decays as $o_P \left( \frac{1}{\sqrt{n}} \right),$ we have that
$\Psi_n - \Psi_0 = (P_n - P_0) \cdot D(P_0) + o_P
\left( \frac{1}{\sqrt{n}} \right),$ which, by a central limit theorem,
establishes a Gaussian limiting distribution for the estimator, with variance
$V(D(P_0))$, the variance of the efficient influence function
when $\Psi$ admits an asymptotically linear representation.

The above implies that $\Psi_n$ is a $\sqrt{n}$-consistent estimator of $\Psi$,
that it is asymptotically normal (as given above), and that it is locally
efficient. This allows us to build Wald-type confidence intervals, where
$\sigma_n^2$ is an estimator of $V(D(P_0))$. The estimator $\sigma_n^2$
may be obtained using the bootstrap or computed directly via
$\sigma_n^2 = \frac{1}{n} \sum_{i = 1}^{n} D^2(\bar{Q}_n^*, g_n)(O_i)$
}

\end{frame}

%%%%%%%%%%%%%%%%%%%%%%%%%%%%%%%%%%%%%%%%%%%%%%%%%%%%%%%%%%%%%%%%%%%%%%%%%%%%%%%%

\begin{frame}[c]{Statistical Inference for TMLEs}

\begin{center}
\begin{itemize}
  \itemsep10pt
  \item Asymptotic distribution of TML estimators has been studied thoroughly:
    $\psi_n - \psi_0 = (P_n - P_0) \cdot D(P_0) + R(\hat{P}^*, P_0)$, giving
    $\psi_n - \psi_0 = (P_n - P_0) \cdot D(P_0) + o_P \left( \frac{1}{\sqrt{n}}
    \right)$.
  \item Have a \textit{Gaussian limiting distribution}
    $\sqrt{n}(\psi_n - \psi) \to N(0, V(D(P_0)))$ when $\psi$ exhibits
    asymptotically linearity.
  \item \textbf{Statistical inference} using Wald-type confidence intervals:
    $\Psi_n \pm z_{\alpha} \cdot \frac{\sigma_n}{\sqrt{n}}$, where $\sigma_n^2$
    is an estimator of $V(D(P_0))$.
  \item Bootstrap for $\sigma_n^2$ or compute directly via
    $\sigma_n^2 = \frac{1}{n} \sum_{i = 1}^{n} D^2(\bar{Q}_n^*, g_n)(O_i)$.
\end{itemize}
\end{center}

\note{
 \begin{enumerate}
   \item If $D(\bar{Q}_n^*, g_n)$ converges to $D(P_0)$ in $L_2(P_0)$ norm.
   \item The size of the class of functions $\bar{Q}_n^*$ and $g_n$ is bounded
     (technically, $\exists \mathcal{F}$ st
     $D(\bar{Q}_n^*, g_n) \in \mathcal{F}$ whp, where $\mathcal{F}$ is a
     Donsker class)
 \end{enumerate}
}

\end{frame}

%%%%%%%%%%%%%%%%%%%%%%%%%%%%%%%%%%%%%%%%%%%%%%%%%%%%%%%%%%%%%%%%%%%%%%%%%%%%%%%%

\begin{frame}[c]{Complication: Multi-Stage Sampling}

\begin{center}
\begin{itemize}
  \itemsep10pt
  \item In the 505 HIV-1 trial, all infected individuals are matched to pairs
    using a complex mechanism.
  \item Using our observed data structure $O = (W,A,Y)$, let us introduce $V =
    (W,Y)$, where $V$ is the set of variables used to define the sampling
    mechanism.
  \item Thus, the observed data structure is now represented $O = (W, \Delta A,
    Y)$ wrt to the full data structure.
    \begin{itemize}
      \item In the above, let $\Delta = f(V)$ be binary st $\Delta \in \{0,
        1\}$.
      \item Further, let $\Pi_0(V) = P(\Delta = 1 \mid V)$ and $\Pi_n(V)$ be an
        estimator of $\Pi_0(V)$.
    \end{itemize}
  \item In this way, our approach accounts for multi-stage sampling (e.g.,
    matched or case-control designs).
\end{itemize}
\end{center}

\note{
}

\end{frame}

%%%%%%%%%%%%%%%%%%%%%%%%%%%%%%%%%%%%%%%%%%%%%%%%%%%%%%%%%%%%%%%%%%%%%%%%%%%%%%%%

\begin{frame}[c]{Multi-Stage Sampling with TMLEs}

\begin{center}
\begin{itemize}
  \itemsep10pt
  \item Rose \& van der Laan (2011) introduce an IPCW-TMLE to be used when the
    data structure takes the form $O = (V, \Delta, \Delta X)$, for multi-stage
    sampling designs.
  \item How? Use an IPC-weighted loss function:
    \[
      \lik(P_X)(O) = \frac{\Delta}{\Pi_n(V)}\lik^F(P_X)(X)
    \]
  \item The IPCW-TMLE solves the full-data efficient influence function (EIF)
    equation:
    \[
      0 = \frac{1}{n}\sum_{i = 1}^n
      \frac{\Delta_i}{\Pi_n(V_i)}D^F(P^*_{X,n})(X_i).
    \]
\end{itemize}
\end{center}

\note{
  Note that the IPCW-TMLE is \textbf{not} fully semiparametric-efficient, unless
  the estimator of $\Pi_n(V)$ is nonparametric, which requires that $V$ be
  discrete. Of course, violation of these assumptions can be highly problematic.
}

\end{frame}

%%%%%%%%%%%%%%%%%%%%%%%%%%%%%%%%%%%%%%%%%%%%%%%%%%%%%%%%%%%%%%%%%%%%%%%%%%%%%%%%

\begin{frame}[c]{Efficiency Under Multi-Stage Sampling}

\begin{center}
\begin{itemize}
  \itemsep10pt
  \item When working in a nonparametric model, it is necessary to use a
    nonparametric estimator of the missingness mechanism to obtain full
    efficiency.
  \item In many practical settings, this further complicates the efficient
    influence function estimating equation
    \[
      \begin{aligned}
      0 =& P_n \frac{\Delta}{\Pi_n^*(V)}D^F(P^*_{X,n}) \\
      &- \left\{\frac{\Delta}{\Pi_n^*(V)} - 1 \right\} \E_n(D^F(P^0_{X,n}) \mid
      \Delta = 1, V).
     \end{aligned}
    \]
\end{itemize}
\end{center}

\note{
}

\end{frame}

%%%%%%%%%%%%%%%%%%%%%%%%%%%%%%%%%%%%%%%%%%%%%%%%%%%%%%%%%%%%%%%%%%%%%%%%%%%%%%%%

\begin{frame}[c]{Putting It Together: Multiple Robustness}

\begin{center}
\begin{itemize}
  \itemsep10pt
  \item We now have a semiparametric-efficient and robust procedure for
    assessing the effect of the intervention $d(a,w) = a + \delta$ even in the
    presence of multi-stage sampling.
  \item Due to the nature of the IPCW-TMLE, we have a form of multiple double
    robustness --- in terms of combinations of $(g, Q)$ and
    $(\Pi, \E_0(D^F(P^F) \mid V))$.
  \item This allows us to assess how simple (additive) shifts of immune response
    variables affect the risk of HIV-1 infection.
\end{itemize}
\end{center}

\note{
}

\end{frame}

%%%%%%%%%%%%%%%%%%%%%%%%%%%%%%%%%%%%%%%%%%%%%%%%%%%%%%%%%%%%%%%%%%%%%%%%%%%%%%%%

\begin{frame}[c]{Software package: R/txshift}

\begin{figure}[H]
  \centering
  \includegraphics[width=\textwidth]{txshift}
  \caption{
    \url{https://github.com/nhejazi/txshift}
  }
\end{figure}

\begin{center}
\begin{itemize}
  \itemsep4pt
  \item Variable importance for continuous interventions.
  \item Take it for a test drive! Coming soon $\ldots$
\end{itemize}
\end{center}

\note{
\begin{itemize}
  \itemsep10pt
  \item Contribute on GitHub: \url{https://github.com/nhejazi/txshift}.
  \item Reach out to us with questions and any feature requests.
\end{itemize}
}

\end{frame}

%%%%%%%%%%%%%%%%%%%%%%%%%%%%%%%%%%%%%%%%%%%%%%%%%%%%%%%%%%%%%%%%%%%%%%%%%%%%%%%%

\begin{frame}[c]{Software Ecosystem: The tlverse!}

\begin{figure}[H]
  \centering
  \includegraphics[width=\textwidth]{tlverse}
  \caption{
    \url{https://github.com/tlverse}
  }
\end{figure}

\begin{center}
\begin{itemize}
  \itemsep4pt
  \item This is a new framework for Targeted Learning with a focus on
    extensibility.
  \item ``txshift'' will be the first of many connector packages ---
    collaboration with Jeremy Coyle and others.
\end{itemize}
\end{center}

\note{
\begin{itemize}
  \itemsep10pt
  \item Contribute on GitHub: \url{https://github.com/nhejazi/txshift}.
  \item Reach out to us with questions and any feature requests.
\end{itemize}
}

\end{frame}

%%%%%%%%%%%%%%%%%%%%%%%%%%%%%%%%%%%%%%%%%%%%%%%%%%%%%%%%%%%%%%%%%%%%%%%%%%%%%%%%

\begin{frame}[c]{Future Work}

\begin{center}
\begin{itemize}
  \itemsep10pt
  \item Exploration of different forms of stochastic treatment shifts --- EH
    Kennedy provides a shift in propensity score space in a recent JASA
    manuscript currently in press (collaboration in progress).
  \item Further refinement of the available software, explore how to provide a
    more efficient and extensible system, including stronger integration with
    the tlverse.
  \item Refinements of statistical theory so as to better work with quantities
    common in survival analysis: hazards? survival?
  \item Assessment of efficacy trials other than the HVTN 505 HIV-1 vaccine
    trial --- perhaps further scientific findings?
\end{itemize}
\end{center}

\note{
}

\end{frame}

%%%%%%%%%%%%%%%%%%%%%%%%%%%%%%%%%%%%%%%%%%%%%%%%%%%%%%%%%%%%%%%%%%%%%%%%%%%%%%%%

\begin{frame}[c]{Review: Summary}

\begin{center}
\begin{itemize}
  \itemsep10pt
  \item The evaluation of vaccine efficacy is a high-impact scientific problem
    that leads to numerous statistical challenges.
  \item Stochastic interventions provide a flexible framework through which
    these statistical problems may be viewed from the perspective of causal
    inference.
  \item Standard targeted minimum loss-based estimation may be augmented to
    handle multi-stage sampling designs, like those common in efficacy trials.
  \item Statistical software is now readily available for deploying these types
    of techniques in a number of settings. We apply these methods in efficacy
    trials.
\end{itemize}
\end{center}

\note{It's always good to include a summary.}

\end{frame}

%%%%%%%%%%%%%%%%%%%%%%%%%%%%%%%%%%%%%%%%%%%%%%%%%%%%%%%%%%%%%%%%%%%%%%%%%%%%%%%%

% don't want dimming with references
\setbeamercovered{}
\beamerdefaultoverlayspecification{}

\begin{frame}[c,allowframebreaks]{References}

\bibliographystyle{apalike}
\nocite{*}
\bibliography{references}

%\note{Here's some work we've talked about. Go check these out if interested.}

\end{frame}

%%%%%%%%%%%%%%%%%%%%%%%%%%%%%%%%%%%%%%%%%%%%%%%%%%%%%%%%%%%%%%%%%%%%%%%%%%%%%%%%

\begin{frame}{Acknowledgments}

\vspace{20pt}

\underline{Collaborators:}

\begin{tabular}{@{}l@{\hspace{1.0cm}}l@{}}
  David C.~Benkeser & \footnotesize \lolit Emory University \\
  Mark J.~van der Laan & \footnotesize \lolit University of California,
    Berkeley \\
  Peter B.~Gilbert & \footnotesize \lolit Fred Hutchinson Cancer Research
    Center \\
  Holly E.~Janes & \footnotesize \lolit Fred Hutchinson Cancer Research
    Center \\
\end{tabular}

\vspace{10mm}

\underline{Funding source}:\\
UC Berkeley NIH BD2K Biomedical Big Data Training Program: T32-LM012417-02

\end{frame}

%%%%%%%%%%%%%%%%%%%%%%%%%%%%%%%%%%%%%%%%%%%%%%%%%%%%%%%%%%%%%%%%%%%%%%%%%%%%%%%%

\begin{frame}[c]{Thank you.}

\Large
Slides: \href{https://goo.gl/LAoDUJ}{goo.gl/LAoDUJ} \quad
\includegraphics[height=5mm]{Figs/cc-zero.png}

\vspace{3mm}
Notes: \href{https://goo.gl/Vq6v5o}{goo.gl/Vq6v5o}

%\vspace{3mm}
%Source (repo): \href{https://goo.gl/m5As73}{goo.gl/m5As73}

\vspace{3mm}
\href{https://www.stat.berkeley.edu/~nhejazi}{\tt
  stat.berkeley.edu/\textasciitilde{}nhejazi}

\vspace{3mm}
\href{https://nimahejazi.org}{\tt nimahejazi.org}

\vspace{3mm}
\href{https://twitter.com/nshejazi}{\tt twitter/@nshejazi}

\vspace{3mm}
\href{https://github.com/nhejazi}{\tt github/nhejazi}

%\note{Here's where you can find me, as well as the slides for this talk.}

\end{frame}

%%%%%%%%%%%%%%%%%%%%%%%%%%%%%%%%%%%%%%%%%%%%%%%%%%%%%%%%%%%%%%%%%%%%%%%%%%%%%%%%

\end{document}

