\documentclass{beamer}
\usetheme[numbering=none]{metropolis}
\def\notescolors{1}
% specifications for presenter mode
%\beamerdefaultoverlayspecification{<+->}
%\setbeamercovered{transparent}

\usepackage[english]{babel}
\usepackage[utf8x]{inputenc}

%\usepackage{coloremoji}
\usepackage{layout}
\usepackage{multirow}
\usepackage{array}
\usepackage{graphicx}
\graphicspath{ {figs/} }

\setbeameroption{hide notes}
\setbeamertemplate{note page}[plain]
\usepackage{listings}
\usepackage{datetime}
\usepackage{url}
\usepackage{tcolorbox}
\usepackage{appendixnumberbeamer}
\usepackage[normalem]{ulem}

\usepackage{tikz}
\def\checkmark{\tikz\fill[scale=0.4](0,.35) -- (.25,0) -- (1,.7) -- (.25,.15) -- cycle;}

% math shorthand
\usepackage{bm}
\usepackage{amsthm}
\usepackage{amstext}
\usepackage[font=small]{caption}
\usepackage{amsmath}
\usepackage{mathtools}
\DeclareMathOperator{\logit}{logit}
\newcommand{\R}{\mathbb{R}}
\newcommand{\D}{\mathcal{D}}
\newcommand{\E}{\mathbb{E}}
\newcommand{\I}{\mathbb{I}}
\newcommand{\pr}{\mathbb{P}}
\newcommand{\F}{\mathcal{F}}
\newcommand{\X}{\mathcal{X}}
\newcommand{\M}{\mathcal{M}}
\newcommand{\lik}{\mathcal{L}}

\newtheorem*{assumption*}{\assumptionnumber}
\providecommand{\assumptionnumber}{}
\makeatletter
\newenvironment{assumption}[2]
 {%
  \renewcommand{\assumptionnumber}{Assumption #1: $\mathcal{#2}$}%
  \begin{assumption*}%
  \protected@edef\@currentlabel{#1: $\mathcal{#2}$}%
 }
 {%
  \end{assumption*}
 }
\makeatother

\DeclarePairedDelimiterX{\infdivx}[2]{(}{)}{%
  #1\;\delimsize\|\;#2%
}
\newcommand{\infdiv}{D\infdivx}
\DeclarePairedDelimiter{\norm}{\lVert}{\rVert}
\DeclareMathOperator*{\argmin}{arg\,min}
\DeclareMathOperator*{\argmax}{arg\,max}

% indepndence notation macro
\newcommand\indep{\protect\mathpalette{\protect\independenT}{\perp}}
\def\independenT#1#2{\mathrel{\rlap{$#1#2$}\mkern2mu{#1#2}}}

% Bibliography
\usepackage{natbib}
\bibpunct{(}{)}{,}{a}{}{;}
%\usepackage{bibentry}

% highlight name in bibliography
\usepackage{xstring}
\def\FormatName#1{%
  \IfSubStr{#1}{Hejazi}{\underline{\textbf{#1}}}{#1}%
}


% title info
\title{\normalsize Characterizing immune correlates of protection in vaccine
  efficacy trials with stochastic-interventional causal effects}
\author{\href{https://nimahejazi.org}{Nima Hejazi}\\[-10pt]}
\institute{
  \begin{figure}[!htb]
    \centering
    \begin{minipage}{0.65\textwidth}
        Department of Biostatistics,\\
        T.H.~Chan School of Public Health,\\
        Harvard University\\[6pt]
        \includegraphics[scale=0.12]{twitter-icon.png}
          \href{https://twitter.com/nshejazi}{nshejazi} \\
        \includegraphics[scale=0.09]{github-icon.png}
          \href{https://github.com/nhejazi}{nhejazi} \\
        \includegraphics[scale=0.12]{homepage.png}
          \href{https://nimahejazi.org}{nimahejazi.org} \\
       Biostatistics Seminar, Boston University\\
     \textit{Joint work with P.B.~Gilbert (Fred Hutch \& UW)}
    \end{minipage}%
    \begin{minipage}{0.3\textwidth}
      \centering
      \vspace{-80pt}
      \includegraphics[width=1.1in]{hsph}
    \end{minipage}
  \end{figure}
}

\date{Thursday, 14\textsuperscript{th} September, 2023}

%%%%%%%%%%%%%%%%%%%%%%%%%%%%%%%%%%%%%%%%%%%%%%%%%%%%%%%%%%%%%%%%%%%%%%%%%%%%%%%%

% Outline at beginning of each section
%\AtBeginSection[]
%{
  %\begin{frame}<beamer>
    %\frametitle{Outline}
    %\tableofcontents[currentsection]
  %\end{frame}
%}

%%%%%%%%%%%%%%%%%%%%%%%%%%%%%%%%%%%%%%%%%%%%%%%%%%%%%%%%%%%%%%%%%%%%%%%%%%%%%%%

\begin{document}

\begin{frame}[noframenumbering]
  \thispagestyle{empty}
  \titlepage

\note{
}

\end{frame}

%%%%%%%%%%%%%%%%%%%%%%%%%%%%%%%%%%%%%%%%%%%%%%%%%%%%%%%%%%%%%%%%%%%%%%%%%%%%%%%%

\begin{frame}[standout]
  Identifying immune correlates of vaccination for HIV-1 and COVID-19
\end{frame}

%%%%%%%%%%%%%%%%%%%%%%%%%%%%%%%%%%%%%%%%%%%%%%%%%%%%%%%%%%%%%%%%%%%%%%%%%%%%%%%%

\begin{frame}[c]{The fights against HIV-1 and COVID-19}

\begin{center}
\begin{itemize}
  \itemsep8pt
  \item The HIV-1 epidemic:
    \begin{itemize}
      \itemsep4pt
      \item 1.5 million new infections occurring annually worldwide;
      \item new infections outpace patients starting antiretroviral therapy;
      %\item $<$40\% efficacy rate for \textit{most efficacious} preventive
        %vaccine.
      \item HIV Vaccine Trials Network's (HVTN) \textit{505} trial evaluated a
        novel antibody boost vaccine~\citep{hammer2013efficacy}.
    \end{itemize}
  \item The COVID-19 \sout{epi}~\sout{pan}~endemic~\citep{antia2021transition}:
    \begin{itemize}
      \itemsep4pt
      \item \sout{270} \sout{331} \sout{619} \sout{643} 770 million total cases
        detected globally;
      \item new variants emerging, with continued formulation of targeted
        vaccines and annual roll-outs expected;
      %\item $\sim$95\% efficacy rate for \textit{most efficacious} vaccine(s).
      \item COVID-19 Prevention Network's (CoVPN) \textit{COVE} trial focused
        on Moderna's (mRNA-1273) vaccine~\citep{baden2021efficacy}.
    \end{itemize}
\end{itemize}
\end{center}

\note{
\begin{itemize}
  \item Baseline covariates($L$): sex, age, BMI, behavioral HIV risk.
  \item Intervention(s) ($A$): post-vaccination T-cell activity markers.
  \item Outcome ($Y$): HIV-1 infection status at week 28 of tiral.
  \item 12-color intracellular cytokine staining (ICS) assay.
  \item Cryopreserved peripheral blood mononuclear cells were stimulated with
    synthetic HIV-1 peptide pools.
  \item A vaccine effective at preventing HIV-1 acquisition would be a
    cost-effective and durable approach to halting the worldwide epidemic.
  \item The study was halted on 22 April 2013 due to absence of vaccine
    efficacy. There was no significant effect of the vaccine on the primary
    infection endpoint of HIV-1 infection between week 28 and month 24.
\end{itemize}

\begin{itemize}
  \item Did we have the time? Polio (7 years), Measles (9 years),
    Chickenpox (34 years), Mumps (4 years), HPV (15 years).
  \item How did we make it? Typical process timeline (73 months) replaced by an
    \textit{accelerated} process of 14 months.
\end{itemize}
}

\end{frame}

%%%%%%%%%%%%%%%%%%%%%%%%%%%%%%%%%%%%%%%%%%%%%%%%%%%%%%%%%%%%%%%%%%%%%%%%%%%%%%%%
\begin{frame}[c]{Evaluating vaccines for HIV-1 and COVID-19}

\begin{center}
\begin{itemize}
  \itemsep8pt
  \item \textit{505}: How would HIV-1 infection risk have differed had
    the boost vaccine modulated antibody responses differently?
  \item \textit{COVE}: How would COVID-19 disease risk have differed for
    alternative vaccine-induced immunogenic response profiles?
  \item \textbf{Question}: How can [HIV-1, COVID-19] vaccines be improved
    through the modulation of immunogenic response profiles?
\end{itemize}
\end{center}

\note{
}

\end{frame}

%%%%%%%%%%%%%%%%%%%%%%%%%%%%%%%%%%%%%%%%%%%%%%%%%%%%%%%%%%%%%%%%%%%%%%%%%%%%%%%%

\begin{frame}[c]{Why measure and analyze immune correlates?}

\begin{center}
\begin{itemize}
  \itemsep8pt
  \item Two, interrelated goals of immune correlates analyses are to
    \begin{itemize}
      \itemsep0pt
      \item identify/validate possible \textit{surrogate
        endpoints}~\citep{prentice1989surrogate};
      \item understand/delineate \textit{protective mechanisms} of vaccines.
    \end{itemize}
  \item If an immune correlate is established to reliably predict VE,
    subsequent efficacy trials may use it as a primary endpoint.
  \item This may accelerate the approval of
    \begin{itemize}
      \itemsep0pt
      \item existing vaccines in different populations (e.g., in children);
      \item new vaccines within the same class (e.g., bivalent mRNA);
      \item inform the development of ``next-generation'' vaccines.
    \end{itemize}
\end{itemize}
\end{center}

\note{
}

\end{frame}

%%%%%%%%%%%%%%%%%%%%%%%%%%%%%%%%%%%%%%%%%%%%%%%%%%%%%%%%%%%%%%%%%%%%%%%%%%%%%%%%

\begin{frame}[c]{Measuring correlates: Two-phase designs}

\begin{center}
\begin{itemize}
  \itemsep8pt
    \item Often, use case-cohort design~\citep{prentice1986case}, a special
      case of two-phase sampling~\citep{breslow2003large}.
    \item Phase 1: measure baseline, vaccination, endpoint on everyone.
    \item Phase 2: given baseline, vaccine, endpoint, select members of
      immune response subcohort with (possibly known) probability.
      \vspace{-1em}
      \begin{itemize}
        \itemsep4pt
        \item \textit{505}: phase-two sample with 100\% of HIV-1 cases and
          matching of non-cases~\citep[$n = 189$ per][]{janes2017higher}.
        \item \textit{COVE}: stratified random subcohort ($n \approx 1600$) and
          all SARS-CoV-2 infection and COVID-19 disease endpoints.
      \end{itemize}
\end{itemize}
\end{center}

\note{
}

\end{frame}

%%%%%%%%%%%%%%%%%%%%%%%%%%%%%%%%%%%%%%%%%%%%%%%%%%%%%%%%%%%%%%%%%%%%%%%%%%%%%%%%

\begin{frame}[c]{A simple two-phase design: Case-cohort}

Assaying >30k samples is expensive, statistically unnecessary.
\vspace{-1em}
\begin{figure}[H]
  \centering
  \includegraphics[scale=0.23]{casecohort}
  \captionsetup{labelformat=empty}
  \vspace{-1.5em}
  \caption{
    Case-cohort design, per~\citet{prentice1986case}, as applied to COVE.
  }
\end{figure}

\note{
}

\end{frame}

%%%%%%%%%%%%%%%%%%%%%%%%%%%%%%%%%%%%%%%%%%%%%%%%%%%%%%%%%%%%%%%%%%%%%%%%%%%%%%%%

\begin{frame}[c]{Two-phase sampling masks the complete data structure}

\begin{center}
\begin{itemize}
  \itemsep8pt
  \item Complete (unobserved) data $X = (L, A, S, Y) \sim P_0^X \in
    \mathcal{M}$:
    \begin{itemize}
      \itemsep4pt
      \item $L$ (baseline covariates): sex, age, BMI, behavioral HIV risk,
      \item $A$ (treatment): randomized assignment to vaccine/placebo,
      \item $S$ (exposure): immune response profile for relevant markers,
      \item $Y$ (outcome of interest): infection status at trial's end.
    \end{itemize}
  \item Observed data $O = (V, B, BX) = (L, A, B, BS, Y) \sim P_0 \in
    \mathcal{M}$.
    \begin{itemize}
      \itemsep4pt
      \item $V \equiv (L, Y)$ are used in defining \textit{outcome-dependent}
        two-phase sampling mechanism $g_{0,B} \coloneqq \pr(B = 1 \mid V)$.
      \item $B \in \{0,1\}$ is an indicator of inclusion in the phase-two
        sample.
      \item $g_{0,B} \coloneqq \pr(B = 1 \mid Y, L)$ must be \textit{known by
        design} or estimated.
      %\item Implicitly conditioning on the vaccine arm: $O = \{X \mid A = 1\}$.
    \end{itemize}
\end{itemize}
\end{center}

\note{
  \begin{itemize}
    \item $P_0^X$ --- true (unknown) distribution of the full data $X$.
    \item $\mathcal{M}$ --- nonparametric statistical model.
  \end{itemize}
}

\end{frame}

%%%%%%%%%%%%%%%%%%%%%%%%%%%%%%%%%%%%%%%%%%%%%%%%%%%%%%%%%%%%%%%%%%%%%%%%%%%%%%%%

\begin{frame}[standout]
  Causal effects for quantitative exposures
\end{frame}

%%%%%%%%%%%%%%%%%%%%%%%%%%%%%%%%%%%%%%%%%%%%%%%%%%%%%%%%%%%%%%%%%%%%%%%%%%%%%%%%

\begin{frame}[c]{Static interventions aren't enough}

\begin{center}
\begin{itemize}
  \itemsep8pt
  \item Describe the manner in which $X$ is hypothetically generated by a
    nonparametric structural equation model~\citep{pearl2009causality}:
    \begin{equation*}
      L = f_L(U_L); A \sim \text{Bern}(0.5);
      S = f_S(A, L, U_S); Y = f_Y(S, A, L, U_Y)
    \end{equation*}
  \item Utility: Implies a model for the distribution of counterfactual RVs
    induced by interventions on the system under study.
  \item A \textit{static} intervention replaces $f_S$ with a specific value $s$
    in its conditional support, i.e., $S \mid L$.
  \item This requires specifying \textit{a priori} a particular value of
    exposure under which to evaluate the outcome --- but what value?
\end{itemize}
\end{center}

\note{
}

\end{frame}

%%%%%%%%%%%%%%%%%%%%%%%%%%%%%%%%%%%%%%%%%%%%%%%%%%%%%%%%%%%%%%%%%%%%%%%%%%%%%%%%

\begin{frame}[c]{Controlled vaccine efficacy (CVE)}

\begin{center}
\begin{itemize}
  \itemsep8pt
  \item For a hypothetical value $s \in \mathcal{S}$, the \textit{controlled
    direct effect} (CDE) quantifies the effect of $A$ on $Y$ while fixing
    $S = s$.
  \item The hypothetical value $S = s$ must be chosen carefully --- to
    be scientifically informative \textit{and} to avoid positivity violations.
  \item For two hypothetical values $s_0, s_1 \in \mathcal{S}$,
    \textit{Controlled Vaccine Efficacy} (CVE)~\citep{gilbert2022controlled} is
  \begin{equation*}
    \text{CVE}(s_0, s_1) = 1 -
      \frac{\E[\pr(Y = 1 \mid S = s_1, A = 1, L = l)]}
      {\E[\pr(Y = 1 \mid \cancel{S = s_0}, A = 0, L = l)]} \ ,
  \end{equation*}
  which contrasts counterfactual risk for vaccine and $S = s_1$ vs.~placebo
  and $S = s_0$, where $s_0 = 0$ by construction.
\end{itemize}
\end{center}

\note{
}

\end{frame}

%%%%%%%%%%%%%%%%%%%%%%%%%%%%%%%%%%%%%%%%%%%%%%%%%%%%%%%%%%%%%%%%%%%%%%%%%%%%%%%%

\begin{frame}[c]{Disease risk under shifted immunogenic responses}

\hspace*{-1cm}\includegraphics[scale=0.41]{shift-1}

\note{
}

\end{frame}

%%%%%%%%%%%%%%%%%%%%%%%%%%%%%%%%%%%%%%%%%%%%%%%%%%%%%%%%%%%%%%%%%%%%%%%%%%%%%%%%

\begin{frame}[c]{Disease risk under shifted immunogenic responses}

\hspace*{-1cm}\includegraphics[scale=0.41]{shift-2}

\note{
}

\end{frame}

%%%%%%%%%%%%%%%%%%%%%%%%%%%%%%%%%%%%%%%%%%%%%%%%%%%%%%%%%%%%%%%%%%%%%%%%%%%%%%%%

\begin{frame}[c]{Disease risk under shifted immunogenic responses}

\hspace*{-1cm}\includegraphics[scale=0.41]{shift-3}

\note{
}

\end{frame}

%%%%%%%%%%%%%%%%%%%%%%%%%%%%%%%%%%%%%%%%%%%%%%%%%%%%%%%%%%%%%%%%%%%%%%%%%%%%%%%%

\begin{frame}[c]{Disease risk under shifted immunogenic responses}

\hspace*{-1cm}\includegraphics[scale=0.41]{shift-4}

\note{
}

\end{frame}

%%%%%%%%%%%%%%%%%%%%%%%%%%%%%%%%%%%%%%%%%%%%%%%%%%%%%%%%%%%%%%%%%%%%%%%%%%%%%%%%

\begin{frame}[c]{Stochastic interventions define the causal effects of shifts}

\begin{center}
\begin{itemize}
  \itemsep8pt
  \item Stochastic interventions modify the value $S$ would naturally assume by
    \textit{shifting} the natural exposure distribution.
  %\item Consider the post-intervention value $S_{\delta} \sim g_{\delta}(\cdot
    %\mid L)$; static interventions are a special case (degenerate distribution).
  \item \cite{diaz2012population, diaz2018stochastic}'s shift
    interventions\footnotemark
     \begin{equation*}\label{shift_intervention}
       d(s, l) =
         \begin{cases}
           s + \delta, & s + \delta < u(l) \quad (\text{if plausible}) \\
           s, & s + \delta \geq u(l) \quad (\text{otherwise})
         \end{cases}
     \end{equation*}
  %\item Such an intervention generates a counterfactual random variable
    %$Y_{g_{\delta}} \coloneqq f_Y(S_{\delta}, L, U_Y)$, with distribution
    %$P_0^{\delta}$, .
  %\item We aim to estimate $\psi_{0,\delta} \coloneqq \E_{P_0^{\delta}}
    %\{Y_{g_{\delta}}\}$, the counterfactual mean under the post-intervention
    %exposure distribution $g_{\delta}$.
  \item Our estimand is $\psi_{0,\delta} \coloneqq
    \E_{P_{\delta,0}}\{Y^{d(S,L)}\}$, which is identified by
    \begin{align*}\label{eqn:identification2012}
      \psi_{0,\delta} = \int_{\mathcal{L}} \int_{\mathcal{S}} &\E_{P_0}
        \{Y \mid S = d(s, l), L = l\} \\ &g_{0, S}(s \mid L = l)
        q_{0, L}(l) d\mu(s)d\nu(l)
    \end{align*}
\end{itemize}
\end{center}

\note{
  \begin{itemize}
    \item For HVTN 505, $\psi_{0,\delta}$ is the counterfactual risk of HIV-1
      infection, had the observed value of the immune response been altered
      under the rule $d(S,L)$ defining $g_{\delta}(\cdot \mid L)$.
  \end{itemize}
}

\footnotetext[1]{\citet{haneuse2013estimation} introduced modified treatment
policies.}

\end{frame}

%%%%%%%%%%%%%%%%%%%%%%%%%%%%%%%%%%%%%%%%%%%%%%%%%%%%%%%%%%%%%%%%%%%%%%%%%%%%%%%%

\begin{frame}[c]{Causal interpretation of statistical target parameter}

\begin{center}
\begin{tcolorbox}
\begin{assumption}{1}{\textit{Stable unit treatment value (SUTVA)}}\label{sutva}
  \begin{itemize}
    \itemsep2pt
    \item $Y^{d(s_i, l_i)}_i$ does not depend on $d(s_j, l_j)$ for
        $i = 1, \ldots, n$ and $j \neq i$, or lack of
        interference~\citep{cox1958planning, rubin1974estimating}
     \item $Y^{d(s, l)} = Y$ in the event $S = d(s, l)$, for $i = 1, \ldots, n$
  \end{itemize}
\end{assumption}
\end{tcolorbox}
\vspace{-0.7em}
\begin{tcolorbox}
\begin{assumption}{2}{\textit{No unmeasured confounding}}\label{ignorability}
  $Y^{d(s, l)} \indep S \mid L = l$, for $i = 1, \ldots, n$
\end{assumption}
\end{tcolorbox}
\vspace{-0.7em}
\begin{tcolorbox}
\begin{assumption}{3}{\textit{Structural positivity}}\label{positivity}
  $s \in \mathcal{S} \implies d(s, l) \in \mathcal{S}$ for all
  $l \in \mathcal{L}$, where $\mathcal{S}$ denotes the support of $S$
  conditional on $L = l$ for all $i = 1, \ldots, n$
\end{assumption}
\end{tcolorbox}
\end{center}

\note{
\begin{itemize}
  \itemsep4pt
  \item This positivity assumption is not the same as that required for
    interventions on binary/categorical exposures.
  \item In particular, we do not require that the intervention density place
    mass across all strata defined by $L$.
  \item Rather, we merely require the post-intervention quantity be seen in the
    observed data for given $s_i \in \mathcal{S}$ and $l_i \in \mathcal{L}$.
\end{itemize}
}

\end{frame}

%%%%%%%%%%%%%%%%%%%%%%%%%%%%%%%%%%%%%%%%%%%%%%%%%%%%%%%%%%%%%%%%%%%%%%%%%%%%%%%%

\begin{frame}[c]{Interpreting the causal effects of shift interventions}

\begin{center}
\begin{itemize}
  \itemsep6pt
  \item Consider a data structure: $(Y_s, s \in \mathcal{S})$.
  \item Let $Y_s = \beta_0 + \beta_1 s + \epsilon_s$, with error $\epsilon_s
    \sim \text{N}(0, \sigma^2_s) \,\, \forall \,\, s \in S$.
  \item For the counterfactual outcomes $(Y_{s' + \delta}, Y_{s'})$, their
    difference $Y_{s' + \delta} - Y_{s'}$ may be expressed (for some
    $s' \in \mathcal{S}$)
    \begin{align*}
      \E Y_{s' + \delta} - \E Y_{s'} &= [\beta_0 + \beta_1 (s' + \delta) +
          \cancel{\E \epsilon_{s' + \delta}}] - [\beta_0 + \beta_1 s' +
          \cancel{\E \epsilon_{s'}}] \\
        &= \cancel{\beta_0 - \beta_0} + \cancel{\beta_1 s' - \beta_1 s'} +
           \beta_1 \delta \\
        &= \beta_1 \delta
    \end{align*}
  \item A \textit{unit shift} for $s' \in S$ (i.e., for $\delta = 1$)
     \textit{causes} a counterfactual difference in $Y$ of magnitude $\beta_1$
     in this simple schematic.
\end{itemize}
\end{center}

\note{
  \begin{itemize}
    \item Note that this analysis is exactly what we're told we \textbf{cannot}
      do in ``linear models 101'' --- that is, the slope of a regression line
      cannot be interpreted as \textit{causing} a change in the outcome.
    \item We extend this result to the mean counterfactual outcomes under the
      nonparametric model $\M$.
  \end{itemize}
}

\end{frame}

%%%%%%%%%%%%%%%%%%%%%%%%%%%%%%%%%%%%%%%%%%%%%%%%%%%%%%%%%%%%%%%%%%%%%%%%%%%%%%%%

\begin{frame}[c]{Stochastic--interventional vaccine efficacy (SVE)}

\begin{center}
\begin{itemize}
  \itemsep8pt
  \item Statistical parameter for vaccine efficacy (VE) estimands:
  \begin{align*}
    \text{SVE}(\delta) &= 1 - \frac{\E[\pr(Y = 1 \mid S = d(s, l), A = 1,
                                    L = l)]}{\pr(Y=1 \mid A=0)} \\
                       &= 1 - \frac{\psi_{0, \delta}}{\pr(Y=1 \mid A=0)}
  \end{align*}
  \item $\pr(Y=1 \mid A=0)$: counterfactual risk in the placebo arm --- i.e.,
     under randomization, $\pr(Y=1 \mid A=0) \equiv \pr(Y(0)=1)$.
  \item Summarizes VE via stochastic interventions across $\delta$, per the
    CoVPN immune correlates SAP\footnotemark~\citep{gilbert2021covpn,
    gilbert2021immune}.
\end{itemize}
\end{center}

\note{
}

\footnotetext[2]{SAP published at
\url{https://doi.org/10.6084/m9.figshare.13198595}.}

\end{frame}

%%%%%%%%%%%%%%%%%%%%%%%%%%%%%%%%%%%%%%%%%%%%%%%%%%%%%%%%%%%%%%%%%%%%%%%%%%%%%%%%

\begin{frame}[standout]
  Efficient estimation in two-phase designs
\end{frame}

%%%%%%%%%%%%%%%%%%%%%%%%%%%%%%%%%%%%%%%%%%%%%%%%%%%%%%%%%%%%%%%%%%%%%%%%%%%%%%%

\begin{frame}{Estimation of the counterfactual mean $\psi_{0,\delta}$}

A \textit{RAL estimator} $\psi_{n,\delta}$ of $\psi_{0,\delta} \coloneqq
\Psi(P_0)$ is \textit{efficient} if and only if
\[
  \psi_{n,\delta} - \psi_{0, \delta} = \frac{1}{n} \sum\limits_{i=1}^n
  D^{\star}(P_0)(O_i) + o_P(n^{-1/2}) \ ,
\]
where $D^{\star}(P)$ is the \textit{efficient influence function} (EIF) of
$\psi_{0,\delta}$ with respect to the nonparametric model $\mathcal{M}$ at a
distribution $P \in \mathcal{M}$.
\vspace{0.25cm}

The EIF of $\psi_{0,\delta}$ is indexed by two key \textit{nuisance parameters}
\begin{align*}
  \overline{Q}_{Y}(S,L) &\coloneqq \E_{P}(Y \mid S, A = 1, L) &
      \mbox{\textit{outcome mechanism}} \\
  g_{S}(S \mid L) &\coloneqq f_P(S \mid A = 1, L) &
    \mbox{\textit{generalized propensity score}}
\end{align*}

\note{
}

\end{frame}

%%%%%%%%%%%%%%%%%%%%%%%%%%%%%%%%%%%%%%%%%%%%%%%%%%%%%%%%%%%%%%%%%%%%%%%%%%%%%%%%

\begin{frame}[c]{Flexible, efficient, doubly robust estimation}

\begin{center}
\begin{itemize}
  \itemsep8pt
  \item The efficient influence function of $\psi_{0, \delta}$ with respect to
    $\mathcal{M}$ is
    \begin{equation*}
      D^{\star}_{F}(P_0)(o) = \frac{g_{0,S}(d^{-1}(s,l) \mid l)}
      {g_{0,S}(s \mid l)} ({y - \overline{Q}_{0,Y}(s,l)}) +
      \overline{Q}_{0,Y}(d(s,l), l) - \psi_{0,\delta}.
    \end{equation*}
  \item The one-step bias-corrected estimator:
    \begin{equation*}\label{tmle}
        \psi_n^{+} = \frac{1}{n} \sum_{i = 1}^n \overline{Q}_{n,Y}(d(S_i, L_i),
        L_i) + D^{\star}_{F,n}(O_i) \,\, .
      \end{equation*}
  \item The TML estimator updates initial estimates of $\overline{Q}_{n,Y}$ to
    $\overline{Q}_{n,Y}^{\star}$ via a tilting procedure that sets
    $\E_P D^{\star}_{F,n}(P_0)(O) \approx 0$:
    \begin{equation*}\label{tmle}
      \psi_n^{\star} = \frac{1}{n} \sum_{i = 1}^n
      \overline{Q}_{n,Y}^{\star}(d(S_i, L_i), L_i) \,\, .
      \end{equation*}
  \item ``Double robust'' -- flexible modeling of nuisance parameters.
\end{itemize}
\end{center}

\note{
  \begin{itemize}
    \item Both estimators are CAN even when nuisance parameters are estimated
      via flexible, machine learning techniques.
    \item Semiparametric-efficient estimation thru solving efficient influence
      function estimating equation wrt the model $\M$.
    \item The auxiliary covariate simplifies when the treatment is in the limits
      (conditional on $W$) --- i.e., for $S_i \in (u(l) - \delta, u(l))$, then
      we have $H(s,l) = \frac{g_0(s - \delta \mid l)}{g_0(s \mid l)} + 1$.
    \item Need to explicitly remind the audience what $u(l)$ is again. It's only
      appeared once at this point, and only been mentioned in passing.
  \end{itemize}
}

\end{frame}

%%%%%%%%%%%%%%%%%%%%%%%%%%%%%%%%%%%%%%%%%%%%%%%%%%%%%%%%%%%%%%%%%%%%%%%%%%%%%%%%

\begin{frame}[c]{Augmented estimators for two-phase sampling designs}

\begin{center}
\begin{itemize}
  \itemsep8pt
  \item \cite{rose2011targeted2sd} suggested inverse probability of censoring
    weighted (IPCW) loss functions:
    \begin{equation*}
      \lik(P_0^X)(O) = \frac{B}{g_{0,B}(Y, L)}\lik(P_0^X)(X)
    \end{equation*}
  \item When the sampling mechanism $g_{0,B}(Y,L)$ is known by design, this
    procedure yields a reasonably reliable estimator.
  \item When data-adaptive regression is warranted --- that is, when
    $g_{0,B}(Y,L)$ is not known by design\footnotemark --- this is inefficient.
\end{itemize}
\end{center}

\note{
}

\footnotetext[3]{Sampling of non-cases in HVTN 505 used
matching~\citep{janes2017higher}.}

\end{frame}

%%%%%%%%%%%%%%%%%%%%%%%%%%%%%%%%%%%%%%%%%%%%%%%%%%%%%%%%%%%%%%%%%%%%%%%%%%%%%%%%

\begin{frame}[c]{Efficiency and multiple robustness~\citep{hejazi2020efficient}}

\begin{center}
\begin{itemize}
  \itemsep8pt
  \item Then, the IPCW augmentation must be applied to the EIF\footnotemark:
    \begin{align*}
      D^{\star}(P_0^X)(o) = &\frac{b}{g_{0,B}(y, l)} D^{\star}_F(P_0^X)(x) -
        \left(1 - \frac{b}{g_{0,B}(y, l)}\right) \\
        &\E(D^{\star}_F(P_0^X)(x) \mid B = 1, Y = y, L = l) \,\, .
    \end{align*}
  \item Expresses observed data EIF $D^{\star}(P_0^X)(o)$ via complete data
     EIF $D^{\star}_F(P_0^X)(x)$; inclusion of second term improves efficiency.
 \item An emergent robustness property --- one each of
    $g_{0,S}(S \mid L)$, $\overline{Q}_{0,Y}(S,L)$ and $g_{0,B}(Y, L)$,
    $\E(D^{\star}_F(P^X_0)(x) \mid B = 1, Y, L)$.
  \item Our \texttt{txshift} \texttt{R} package implements our estimators of
    $\psi_{0,\delta}$.
\end{itemize}
\end{center}

\note{
\begin{itemize}
  \itemsep6pt
  \item The expectation of the full data EIF $D^{\star}_F(P_0^X)(x)$, taken
    only over units selected by the sampling mechanism (i.e., $B = 1$).
\end{itemize}
}

\footnotetext[4]{\citet{robins1994estimation} explored a similar correction for
related sampling designs.}

\end{frame}

%%%%%%%%%%%%%%%%%%%%%%%%%%%%%%%%%%%%%%%%%%%%%%%%%%%%%%%%%%%%%%%%%%%%%%%%%%%%%%%%


\begin{frame}[c]{Comparing reweighted and augmented estimators}

\hspace*{-0.75cm}\includegraphics[scale=0.255]{simple_effect_panel_delta_upshift}

\note{
}

\end{frame}

%%%%%%%%%%%%%%%%%%%%%%%%%%%%%%%%%%%%%%%%%%%%%%%%%%%%%%%%%%%%%%%%%%%%%%%%%%%%%%%

\begin{frame}[standout]
  Predicting and bridging vaccine efficacy
\end{frame}

%%%%%%%%%%%%%%%%%%%%%%%%%%%%%%%%%%%%%%%%%%%%%%%%%%%%%%%%%%%%%%%%%%%%%%%%%%%%%%%%

\begin{frame}[c]{SVE \textit{prediction} of HIV-1 risk in the HVTN 505 trial}

\vspace{-0.3in}
\begin{figure}[H]
  \centering
  \includegraphics[scale=0.235]{cd8_msm_tmle_summary}
  \captionsetup{labelformat=empty}
  \vspace{-1.5em}
  \caption{
    HIV-1 risk change across CD8+ score (\texttt{txshift} \texttt{R}
    package).
  }
\end{figure}

\note{
}

\end{frame}

%%%%%%%%%%%%%%%%%%%%%%%%%%%%%%%%%%%%%%%%%%%%%%%%%%%%%%%%%%%%%%%%%%%%%%%%%%%%%%%%

\begin{frame}[c]{SVE \textit{prediction} of mRNA-1273 VE in the CoVPN COVE
trial}

\vspace*{-0.1cm}
\hspace*{-0.5cm}\includegraphics[scale=0.24]{sve_pseudoneutid50_simplified_talksonly}

\note{
}

\end{frame}

%%%%%%%%%%%%%%%%%%%%%%%%%%%%%%%%%%%%%%%%%%%%%%%%%%%%%%%%%%%%%%%%%%%%%%%%%%%%%%%%

\begin{frame}[c]{Pooled phase 1 studies: PsV nAb responses across variants}

\vspace*{0.2cm}
\hspace*{-0.8cm}\includegraphics[scale=0.31]{variant_id50_eda}

\note{
}

\end{frame}

%%%%%%%%%%%%%%%%%%%%%%%%%%%%%%%%%%%%%%%%%%%%%%%%%%%%%%%%%%%%%%%%%%%%%%%%%%%%%%%%

\begin{frame}[c]{SVE \textit{bridging} of mRNA-1273
VE~\citep{hejazi2023stochastic}}

\vspace*{-0.1cm}
\hspace*{-0.3cm}\includegraphics[scale=0.21]{sve_pseudoneutid50_immunobridging_2dose}

\note{
SVE is being measured for COVID-19 from 7-100 days post-4 weeks following
vaccination (Day 57 for 2 doses, Day 29 for 3 doses). Note that it is studying
the exact same data and time period as was studied in the \textit{Science}
paper.
}

\end{frame}

%%%%%%%%%%%%%%%%%%%%%%%%%%%%%%%%%%%%%%%%%%%%%%%%%%%%%%%%%%%%%%%%%%%%%%%%%%%%%%%%

%\begin{frame}[c]{SVE Predictions vs.~Real-World Reported Estimates}

%\begin{center}
%\begin{itemize}
  %\itemsep6pt
  %\item Compared $\delta$-calibrated SVE predictions to reported VE estimates,
      %from TND studies or RCTs.
  %\item Inclusion/exclusion criteria for TND-based VE estimates\footnotemark:
    %\begin{itemize}
      %\itemsep2pt
      %\item VE estimated by direct measurement of SARS-CoV-2 variants.
      %\item Reported VE estimates for mRNA vaccines (BNT162b2 or mRNA-1273),
        %studying VE 2--6 months post-2\textsuperscript{nd} dose.
      %\item Allowed flexibility in choice of dosing interval, with some studies
        %extending to 12 weeks between doses.
    %\end{itemize}
  %\item Studied concordance of SVE predictions and reported (TND or RCT)
      %estimates of VE following most recent vaccine dose.
%\end{itemize}
%\end{center}

%\footnotetext[5]{\scriptsize Comparison of TND studies performed in
%collaboration with Dr.~Lindsay Carpp.}

%\note{
%}

%\end{frame}

%%%%%%%%%%%%%%%%%%%%%%%%%%%%%%%%%%%%%%%%%%%%%%%%%%%%%%%%%%%%%%%%%%%%%%%%%%%%%%%%

%\begin{frame}[c]{Comparison of SVE Predictions and Reported VE Estimates}

%\hspace*{-0.5cm}\includegraphics[scale=0.285]{tnd_sve_calib_pfizer_rescaled}

%\note{
%}

%\end{frame}

%%%%%%%%%%%%%%%%%%%%%%%%%%%%%%%%%%%%%%%%%%%%%%%%%%%%%%%%%%%%%%%%%%%%%%%%%%%%%%%%

%\begin{frame}[c]{Concordance of SVE Predictions and Reported VE Estimates}

%\hspace*{-0.5cm}\includegraphics[scale=0.285]{tnd_sve_scatter_2dose}

%\note{
%}

%\end{frame}

%%%%%%%%%%%%%%%%%%%%%%%%%%%%%%%%%%%%%%%%%%%%%%%%%%%%%%%%%%%%%%%%%%%%%%%%%%%%%%%%

%\begin{frame}[c]{Summary of SVE Prediction for Immunobridging}

%\begin{center}
%\begin{itemize}
  %\itemsep6pt
  %\item SVE prediction shows sharp changes in VE with shifts to the GM titer of
    %the PsV nAb correlate in vaccinees.
    %%\begin{itemize}
      %%\itemsep2pt
      %%\item Positive $\delta$: saturates at $\approx$100\%.
      %%\item Negative $\delta$: large range, from -20\%--92\%.
    %%\end{itemize}
  %\item Bridging VE across variants indicates VE drops but stabilizes at 50\%,
    %if the model based on ancestral D614G strain holds.
    %%\begin{itemize}
      %%\itemsep2pt
  %\item Post-2\textsuperscript{nd} dose: For most variants (excepting
    %omicron), the VE estimate ranges from 50\% (mu) to 80\% (epsilon).
      %%\item Post-3\textsuperscript{rd} dose: For omicron lineages, VE estimate
        %%lies within 80\%--92\% across five subvariants (lowest VE
        %%vs.~BA.2.12.1).
    %%\end{itemize}
  %\item SVE predictions and real-world VE estimates well-correlated, but
      %SVE predictions may be underestimates, as PsV nAb correlate is an
      %\textit{imperfect} causal \textit{mediator} of total VE.
    %%\begin{itemize}
      %%\itemsep2pt
      %%\item TND studies are prone to bias, tending to overestimate VE.
      %%\item SVE predictions may be underestimates, since the PsV nAb correlate
        %%is an \textit{imperfect} causal \textit{mediator} of the total VE.
    %%\end{itemize}
%\end{itemize}
%\end{center}

%\note{
%}

%\end{frame}

%%%%%%%%%%%%%%%%%%%%%%%%%%%%%%%%%%%%%%%%%%%%%%%%%%%%%%%%%%%%%%%%%%%%%%%%%%%%%%%%

%\begin{frame}[standout]
  %Zooming Out
%\end{frame}

%%%%%%%%%%%%%%%%%%%%%%%%%%%%%%%%%%%%%%%%%%%%%%%%%%%%%%%%%%%%%%%%%%%%%%%%%%%%%%%

%\begin{frame}[c]{Going ``Off-Road'': Real-World Complexities}

%\begin{center}
%\begin{itemize}
  %\itemsep8pt
  %\item We considered the case of $O = (L, A, BS, Y, B)$, but what about $O =
    %(V, L, A, BS, Y, B)$ or $O = (L, A, Z, BS, Y, B)$?
    %\begin{itemize}
      %\item $Z$: \textit{unmeasured} baseline confounder (e.g., prior
        %infection)
      %\item $A \in \{0, 1\}$: randomized treatment assignment
      %\item $Z$: post-treatment confounder (e.g., unblinded risky
        %behavior)
      %\item $S$: candidate immune correlates (causal mediators)
      %\item $Y$: symptomatic SARS-CoV-2 (or HIV-1) infection
      %\item $B \coloneqq f(Y, L)$: selection into two-phase sample
    %\end{itemize}
  %\item And what about survival endpoints, $O = (L, A, BS, \Delta,
    %\widetilde{T}, B)$?
    %\begin{itemize}
      %\item $\widetilde{T} = \min(T_F, T_C)$: possibly right-censored time
        %to failure
      %\item $\Delta = \mathbb{I}(T_F < T_C)$: indicator of failure
        %endpoint occurrence
      %\item Could making $B$ a function of $\widetilde{T}$ improve sampling
        %efficiency?
    %\end{itemize}
%\end{itemize}
%\end{center}

%\note{
  %\begin{itemize}
  %\item Goal: assess \textit{indirect} effect of vaccination through mCoPs.
  %\item Define/identify new mCoPs to be used as surrogate endpoints.
  %\item Could also have missing outcome in the binary endpoint case.
  %\end{itemize}
%}

%\end{frame}

%%%%%%%%%%%%%%%%%%%%%%%%%%%%%%%%%%%%%%%%%%%%%%%%%%%%%%%%%%%%%%%%%%%%%%%%%%%%%%%%%

\begin{frame}[c]{The Big Picture}

\begin{center}
\begin{itemize}
  \itemsep8pt
  \item Flexible stochastic interventions help to formulate novel modified
    treatment policies (based on ``natural'' treatment conditions).
  \item \textit{Modified treatment policies} address causal questions about
    \textit{realistic} manipulations of quantitative intervention variables.
  \item Large-scale vaccine trials routinely use two-phase designs --- but need
    to adjust (carefully!) for resultant sampling bias.
  \item Efficient estimators with double/multiple robustness can safely answer
    such questions \textit{while} incorporating machine learning.
  \item Open source software for such statistical analyses is critical for the
    methods to have any impact on real-world studies.
\end{itemize}
\end{center}

\note{
}

\end{frame}

%%%%%%%%%%%%%%%%%%%%%%%%%%%%%%%%%%%%%%%%%%%%%%%%%%%%%%%%%%%%%%%%%%%%%%%%%%%%%%%%

\begin{frame}[c]{Thank you!}

\Large{Thanks for listening. Any questions?}

\vspace{2mm}
\includegraphics[scale=0.14]{homepage.png} \url{https://nimahejazi.org}

\vspace{2mm}
\includegraphics[scale=0.11]{github-icon.png}
  \url{https://github.com/nhejazi}

\vspace{2mm}
\includegraphics[scale=0.14]{twitter-icon.png}
  \url{https://twitter.com/nshejazi}

\vspace{2mm}
\includegraphics[scale=0.14]{paper-icon.png}
  \url{https://doi.org/10.1111/biom.13375}

\vspace{2mm}
\includegraphics[scale=0.14]{paper-icon.png}
  \citet{hejazi2023stochastic} to appear in \textit{IJID}

\end{frame}

%%%%%%%%%%%%%%%%%%%%%%%%%%%%%%%%%%%%%%%%%%%%%%%%%%%%%%%%%%%%%%%%%%%%%%%%%%%%%%%

\appendix
\begin{frame}[standout]
  Appendix
\end{frame}

%%%%%%%%%%%%%%%%%%%%%%%%%%%%%%%%%%%%%%%%%%%%%%%%%%%%%%%%%%%%%%%%%%%%%%%%%%%%%%%%

\begin{frame}[c]{Immune correlates of
  protection~\citep{plotkin2012nomenclature}}

\begin{center}
\begin{itemize}
  \itemsep8pt
  \item Correlate of Protection (CoP): immune marker statistically predictive
    of vaccine efficacy, not necessarily mechanistic.
  \item Mechanistic CoP (mCoP): immune marker that is causally and
    mechanistically responsible for protection.
  \item Nonmechanistic CoP (nCoP): immune marker that is predictive but not a
    causal agent of protection.
  \item A CoP is a \textit{candidate surrogate}
      endpoint~\citep{prentice1989surrogate} --- primary
      endpoint in future trials if reliably predictive.
\end{itemize}
\end{center}

\note{
}

\end{frame}

%%%%%%%%%%%%%%%%%%%%%%%%%%%%%%%%%%%%%%%%%%%%%%%%%%%%%%%%%%%%%%%%%%%%%%%%%%%%%%%%

\begin{frame}[c]{Literature: \cite{diaz2012population, diaz2018stochastic}}

\begin{center}
\begin{itemize}
  \itemsep8pt
  \item \textit{Proposal:} Evaluate outcome under an altered
    \textit{intervention distribution} --- e.g.,
    $P_{\delta}(g_{0,S})(S = s \mid L) = g_{0,S}(s - \delta(L) \mid L)$.
  \item Identification conditions for a statistical parameter of the
    counterfactual outcome $\psi_{0,\delta}$ under such an intervention.
  \item Show that the causal quantity of interest $\E_{P_0^{\delta}}
    \{Y_{d(S, L)}\}$ is identified by a functional of the distribution of $O$,
    i.e.,
    \begin{align*}\label{eqn:identification2012}
      \psi_{0,\delta} = \int_{\mathcal{L}} \int_{\mathcal{S}} &\E_{P_0}
        \{Y \mid S = d(s, l), L = l\} \\ &g_{0, S}(s \mid L = l) \cdot
        q_{0, L}(l) d\mu(s)d\nu(l)
    \end{align*}
\end{itemize}
\end{center}

\note{
  \begin{itemize}
    \item The identification result allows us to write down the causal quantity
      of interest in terms of a functional of the observed data.
    \item Key innovation: loosening standard assumptions through a change in
      the observed intervention mechanism.
    \item Problem: globally altering an intervention mechanism does not
      necessarily respect individual characteristics.
    \item The authors build IPW, one-step, and TML estimators, comparing the
      three different approaches.
  \end{itemize}
}

\end{frame}

%%%%%%%%%%%%%%%%%%%%%%%%%%%%%%%%%%%%%%%%%%%%%%%%%%%%%%%%%%%%%%%%%%%%%%%%%%%%%%%%

\begin{frame}[c]{Literature: \cite{haneuse2013estimation}}

\begin{center}
\begin{itemize}
  \itemsep8pt
  \item \textit{Proposal:} Characterization of stochastic interventions as
    \textit{modified treatment policies} (MTPs).
  \item Assumption of \textit{piecewise smooth invertibility} allows for the
    post-intervention distribution of any MTP to be recovered:
    \begin{equation*}
      g_{0, S}(s \mid l; \delta) = \sum_{j = 1}^{J(l)} \mathbb{I}_{\delta, j}
      \{h_j(s, l), l\} g_{0,S}\{h_j(s, l) \mid l\} h^{'}_j(s, l)
    \end{equation*}
  \item MTPs account for the natural value of exposure $S$ yet may be
    interpreted as imposing an altered intervention mechanism.
\end{itemize}
\end{center}

\note{
  \begin{itemize}
    \item Shifts of the form $\delta(S, L)$ are considerably more interesting
      since these are realistic intervention policies.
    \item Example: consider an individual with an extremely high immune response
      but whose baseline covariates $L$ suggest we shift the response still
      higher. Such a shift may not be biologically plausible (impossible, even)
      but we cannot account for this if the shift is only a function of $L$.
    \item The authors build IPW, outcome regression, and non-iterative doubly
      robust estimators, as well as an approach based on MSMs.
    \item Piecewise smooth invertibility: This assumption ensures that we can
      use the change of variable formula when computing integrals over $S$ and
      it is useful to study the estimators that we propose in this paper.
  \end{itemize}
}

\end{frame}

%%%%%%%%%%%%%%%%%%%%%%%%%%%%%%%%%%%%%%%%%%%%%%%%%%%%%%%%%%%%%%%%%%%%%%%%%%%%%%%%

%\begin{frame}[c]{Literature: \cite{young2014identification}}

%\begin{center}
%\begin{itemize}
  %\itemsep8pt
  %\item Establishes equivalence between g-formula when proposed intervention
    %depends on natural value and when it does not.
  %\item This equivalence leads to a sufficient positivity condition for
    %estimating the counterfactual mean under MTPs via the same statistical
    %functional studied in \cite{diaz2012population}.
  %\item Extends earlier identification results, providing a way to use the same
    %statistical functional to assess $\E Y_{d(S,L)}$ or $\E Y_{d(L)}$.
  %\item The authors also consider limits on implementing shifts $d(S,L)$, and
    %address working in a longitudinal setting.
%\end{itemize}
%\end{center}

%\note{
%}

%\end{frame}

%%%%%%%%%%%%%%%%%%%%%%%%%%%%%%%%%%%%%%%%%%%%%%%%%%%%%%%%%%%%%%%%%%%%%%%%%%%%%%%%

\begin{frame}[c]{Slope in a semiparametric model}

\begin{center}
\begin{itemize}
  \itemsep8pt
  \item Consider the stochastic intervention $g_{\delta}(\cdot \mid L)$:
    \begin{align*}
      \E Y_{g_{\delta}} &= \int_L \int_s \E(Y \mid S = s, L) g(s - \delta
            \mid L) ds dP_0(L) \\
        &= \int_L \int_z \E(Y \mid S = z + \delta, L) g(z \mid L) dz dP_0(L),
    \end{align*}
      defning the change of variable $z = s - \delta$.
  \item For a semiparametric model, $\E (Y \mid S = z, L) = \beta z +
    \theta(L)$:
    \begin{align*}
      \E Y_{g_{\delta}} - \E Y &= \int_L \int_z
      \begin{aligned}[t]
        & [\E(Y \mid S = z + \delta, L) - \E(Y \mid S = z, L)] \\
        & g(z \mid L) dz dP_0(L)
      \end{aligned} \\
      &= [\beta (z + \delta) + \theta(L)] - [\beta z + \theta(L)] \\
      &= \beta \delta
    \end{align*}
\end{itemize}
\end{center}

\note{
}

\end{frame}

%%%%%%%%%%%%%%%%%%%%%%%%%%%%%%%%%%%%%%%%%%%%%%%%%%%%%%%%%%%%%%%%%%%%%%%%%%%%%%%%

\begin{frame}[c]{Flexible conditional density estimation of $g_{0,S}$}

\begin{center}
\begin{itemize}
  \itemsep8pt
  \item \cite{diaz2011super}'s conditional density estimator:
    \begin{equation*}
      g_{n, \alpha}(s \mid L) = \frac{\pr (s \in [\alpha_{t-1}, \alpha_t)
        \mid L)}{\alpha_t - \alpha_{t-1}}.
    \end{equation*}
    %for $\alpha_{t-1} \leq s < \alpha_t$.
    \vspace{-0.5em}
    \begin{itemize}
      \itemsep4pt
      \item Re-expressed as hazard regressions in repeated measures data.
      \item Tuning parameter $t$ $\approx$ bandwidth in kernel density
        estimation.
    \end{itemize}
  \item When c\`{a}dl\`{a}g (RCLL) with finite sectional variation, we have
    {\small{
    \begin{equation*}
     \logit \{\pr(s \in [\alpha_{t-1}, \alpha_t) \mid L)\} = \beta_0 +
       \sum_{w \subset\{1,\ldots,d\}} \sum_{i=1}^{n} \beta_{w,i} \phi_{w,i},
    \end{equation*} }
    }
    for appropriate basis functions
    $\{ \phi_{w,i} \}_{i=1}^n$~\citep{gill1995inefficient}.
\end{itemize}
\end{center}

\note{
}

\end{frame}

%%%%%%%%%%%%%%%%%%%%%%%%%%%%%%%%%%%%%%%%%%%%%%%%%%%%%%%%%%%%%%%%%%%%%%%%%%%%%%%

\begin{frame}[c]{Flexible conditional density estimation of $g_{0,S}$}

\begin{center}
\begin{itemize}
  \itemsep8pt
  \item Utilizing a particular basis construction for $\phi_w$,
    \citet{vdl2017generally}'s HAL estimator achieves $n^{-1/4}$
    convergence rate\footnotemark.
  \item Loss-based cross-validation allows selection of a suitable HAL
    estimator, which has only the $\ell_1$ regularization term $\lambda$:
    {\small{
    \begin{equation*}
      \beta_{n, \lambda}= \argmin_{\beta: \lvert \beta_0 \rvert + \sum_{w
        \subset\{1,\ldots,d\}} \sum_{i=1}^{n} \lvert \beta_{w,i} \rvert <
        \lambda} P_n \lik(g_{\beta,\lambda,S}),
    \end{equation*} }
    }
    where $\lik(\cdot)$ is an appropriate loss function, giving
    $\{\lambda_n, \beta_n\}$.
  \item We denote by $g_{n,\lambda,S} \coloneqq g_{\beta_{n, \lambda},S}$, the
    HAL estimate of $g_{0,S}$.
  \item Our \texttt{haldensify} \texttt{R} package implements our estimator of
    $g_{0,S}$.
\end{itemize}
\end{center}

\note{
\begin{itemize}
  \item C\`{a}dl\`{a}g: right-hand continuous with left-hand limits
  \item Improved convergence rate~\citep{bibaut2019fast}:
    \begin{equation*}
      \lvert \theta_{n,M_n} - \theta_{0} \rvert_{P_0} =
      o_P(n^{-1/3} \log(n)^{d/2}) \ .
    \end{equation*}
\end{itemize}
}

\footnotetext[6]{Similar rates can be achieved via \textit{local} (vs.~global)
smoothness assumptions on $g_{n,S}$~\citep[see, e.g.,][]{robins2008higher,
mukherjee2017semiparametric, liu2021adaptive}.
}

\end{frame}

%%%%%%%%%%%%%%%%%%%%%%%%%%%%%%%%%%%%%%%%%%%%%%%%%%%%%%%%%%%%%%%%%%%%%%%%%%%%%%%

\begin{frame}{A useful class of functions}
\vspace{0.75cm}

Consider space of \textit{cadlag} functions with \textit{finite variation
norm}.

\textbf{Def.} cadlag = \textit{left-hand continuous} with
\textit{right-hand limits}
\vspace{0.25cm}

\textbf{Variation norm}
Let $\theta_s(u)=\theta(u_s,0_{s^c})$ be the \textit{section} of $\theta$ that
sets the coordinates in $s$ \textit{equal to zero}.
\vspace{0.15cm}

The \textit{variation norm} of $\theta$ can be written:
\[
  \lvert \theta \rvert_v=\sum_{s\subset\{1,\dots,d\}}\int \mid
  d\theta_s(u_s)\mid ,
\]
where $x_s=(x(j):j\in s)$ and the sum is over all subsets.

\note{
}

\end{frame}

%%%%%%%%%%%%%%%%%%%%%%%%%%%%%%%%%%%%%%%%%%%%%%%%%%%%%%%%%%%%%%%%%%%%%%%%%%%%%%%

\begin{frame}{Sectional variation norm}

We can represent the function $\theta$ as
\begin{eqnarray*}
\theta(x)&=&\theta(0) + \sum_{s\subset\{1,\ldots,d\}}
  \int \I(x_s\geq u_s)d\theta_s(u_s),
\end{eqnarray*}

For discrete measures $d\theta_s$ with \textit{support points} $\{u_{s,j}:j\}$
we get a \textit{linear combination} of indicator \textit{basis functions}:
\[
  \theta(x) = \theta(0) + \sum_{s\subset\{1,\ldots,d\}}\sum_{j} \beta_{s,j}
    \theta_{u_{s,j}}(x),
\]
where $\beta_{s,j}=d\theta_s(u_{s,j})$,
$\theta_{u_{s,j}}(x) = \I(x_s \ge u_{s,j})$,
and
\begin{equation*}
\lvert \theta \rvert_v = \theta(0) + \sum_{s\subset\{1,\ldots,d\}}
  \sum_j \lvert \beta_{s,j} \rvert.
\end{equation*}

\note{
}

\end{frame}

%%%%%%%%%%%%%%%%%%%%%%%%%%%%%%%%%%%%%%%%%%%%%%%%%%%%%%%%%%%%%%%%%%%%%%%%%%%%%%%

\begin{frame}{Convergence rate for HAL estimators}
\vspace{0.25cm}

We have, for $\alpha(d)=1/(d+1)$,
\[
  \lvert \theta_{n,M} - \theta_{0,M} \rvert_{P_0}=o_P(n^{-(1/4+\alpha(d)/8)}).
\]

Thus, if we select $M> \lvert \theta_0 \rvert_v$, then
\[
  \lvert \theta_{n,M} - \theta_{0} \rvert_{P_0} =
  o_P(n^{-(1/4+\alpha(d)/8)})\ .
\]

Due to oracle inequality for the cross-validation selector $M_n$,
\[
  \lvert \theta_{n,M_n} - \theta_{0} \rvert_{P_0} =
  o_P(n^{-(1/4+\alpha(d)/8)}) \ .
\]

Improved convergence rate~\citep{bibaut2019fast}:
\[
  \lvert \theta_{n,M_n} - \theta_{0} \rvert_{P_0} =
  o_P(n^{-1/3} \log(n)^{d/2}) \ .
\]

\note{
  \begin{itemize}
  \item The goal was to construct nuisance parameter estimators that are
    \textit{consistent} and \textit{converge faster} than $n^{-1/4}$ under
    \textit{minimal assumptions}.
  \item We have assumed enough \textit{smoothness} that HAL will
    \textit{converge faster} than $n^{-1/4}$, but retain enough flexibility
    that \textit{consistency} is also preserved.
  \end{itemize}
}

\end{frame}

%%%%%%%%%%%%%%%%%%%%%%%%%%%%%%%%%%%%%%%%%%%%%%%%%%%%%%%%%%%%%%%%%%%%%%%%%%%%%%%%

\begin{frame}[c]{Algorithm for TML estimation}

\begin{center}
\begin{enumerate}\label{tmle_algo}
  \itemsep6pt
  \item Construct initial estimators $g_{n,S}$ of $g_{0,S}(S, L)$ and $Q_{n,Y}$
    of $\overline{Q}_{0,Y}(S, L)$, perhaps using data-adaptive regression
    techniques.
  \item For each observation $i$, compute an estimate $H_n(s_i, l_i)$ of the
    auxiliary covariate $H_0(s_i, l_i)$.
  \item Estimate the parameter $\epsilon$ in the logistic regression model
    $$\text{logit}\overline{Q}_{\epsilon, n, Y}(s, l) =
    \text{logit}\overline{Q}_{n,Y}(s, l) + \epsilon H_n(s, l),$$
    or an alternative regression model incorporating weights.
  \item Compute TML estimator $\Psi_n$ of the target parameter, defining the
    update $\overline{Q}_{n,Y}^{\star}$ of the initial estimate
    $\overline{Q}_{n,Y}$ to $\overline{Q}_{\epsilon_n, n, Y}$:
    \begin{equation*}\label{tmle}
      \Psi_n = \Psi(P_n^{\star}) = \frac{1}{n} \sum_{i = 1}^n
        \overline{Q}_{n,Y}^{\star}(d(S_i, L_i), L_i).
      \end{equation*}
\end{enumerate}
\end{center}

\note{
  \begin{itemize}
    \item We recommend using nonparametric methods for the initial estimators,
      as consistent estimation is necessary for efficiency of the estimator
      $\Psi_n$.
    \item Intuition for the submodel fluctuation?
  \end{itemize}
}

\end{frame}

%%%%%%%%%%%%%%%%%%%%%%%%%%%%%%%%%%%%%%%%%%%%%%%%%%%%%%%%%%%%%%%%%%%%%%%%%%%%%%%%

\begin{frame}[c]{Algorithm for IPCW-TML estimation}

\begin{center}
\begin{enumerate}\label{ipcwtmle_algo}
  \itemsep8pt
  \item Using all observed units, estimate the sampling mechanism
    $g_{0,B}(Y, L)$, perhaps using data-adaptive regression methods.
  \item Using only observed units in the phase-two sample $B = 1$,
    construct initial estimators $g_{n,S}(S,L)$ and $\overline{Q}_{n,Y}(S,L)$,
    weighting by the sampling mechanism estimate $g_{n,B}(Y,L)$.
  \item With the approach described for the full data case, compute
    $H_n(s_i,l_i)$, and fluctuate submodel via logistic regression.
  \item Compute IPCW-TML estimator $\Psi_n$ of the target parameter, by solving
    the IPCW-augmented EIF estimating equation.
  \item Iteratively update estimated sampling weights $g_{n,B}(Y,L)$ and
    IPCW-augmented EIF, updating TMLE in each iteration.
\end{enumerate}
\end{center}

\note{
  \begin{itemize}
    \item We recommend using nonparametric methods for the initial estimators,
      as consistent estimation is necessary for efficiency of the estimator
      $\Psi_n$.
    \item Intuition for the submodel fluctuation?
    \item This process includes the use of HAL to fit the regression of the EIF
      contributions on the sampling node $\{Y, L\}$.
  \end{itemize}
}

\end{frame}

%%%%%%%%%%%%%%%%%%%%%%%%%%%%%%%%%%%%%%%%%%%%%%%%%%%%%%%%%%%%%%%%%%%%%%%%%%%%%%%%

% don't want dimming with references
\setbeamercovered{}
\beamerdefaultoverlayspecification{}

\begin{frame}[c,allowframebreaks]{}

\tiny
\bibliographystyle{apalike}
%\nocite{*}
\bibliography{references}
%\itemize

\end{frame}
\end{document}
